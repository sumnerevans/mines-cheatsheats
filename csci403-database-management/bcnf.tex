\documentclass{../cheatsheet}

\course{CSCI 406}
\cheatsheetname{Boyce-Codd Normal Form}

\begin{document}

\begin{multicols*}{2}
\section{Definition}
A relation $R$ is a Boyce-Codd Normal Form (BCNF) if for
every non-trivial functional dependency $X \rightarrow A$ on $R$, $X$ is a
superkey of $R$.

\section{Example}
Consider our relation schema in Figure 1.

One of the (non-trivial) functional dependencies we identified was:

\begin{center}
    instructor \textrightarrow office
\end{center}

Clearly, instructor is not a superkey of the relation.

Therefore, we say that the FD instructor \textrightarrow office
\textit{violates} BCNF, and the relation schema is not BCNF.

\section{Decomposition Algorithm}
while some relation schema is not in BCNF: \\
\tab choose some relation schema $R$ not in BCNF \\
\tab choose some FD $X \rightarrow Y$ on $R$ that violates BCNF \\
\tab (optional) expand $Y$ so that $Y = X^{+}$ (closure of $X$) \\
\tab let $Z$ be all attributes of $R$ not included in $X$ or $Y$ \\
\tab replace $R$ with two new relations \\
\tab\tab $R_1$, containing attributes ${X, Y}$ \\
\tab\tab $R_2$, containing attributes ${X, Z}$

Note, this algorithm is \textbf{\textit{not}} deterministic - you can decompose
differently if you choose differently

\section{Decomposition Notes}
\begin{itemize}
    \item The final step above is accomplished simply by projection onto the
        attributes in $R_1$ and $R_2$. (Recall that this may result in fewer
        tuples.)

    \item After decomposing, you will need to establish which FDs now apply to
        $R_1$ and $R_2$ as well as determine their superkeys, in order to
        determine if they are now in BCFN.

    \item The optional step of expanding $Y$ is recommended, as it tends to
        result in fewer larger relation schemas, and may reduce the problem
        faster - e.g. consider decomposing instructor \textrightarrow office.
\end{itemize}

\section{Decomposition Example}
Let's use the relation schema in Figure 1 as an example.

For this schema, we listed the following FDs:

\begin{itemize}
    \item instructor \textrightarrow office (violates BCNF)
    \item instructor \textrightarrow email (violates BCNF)
    \item \{course id, section\} \textrightarrow instructor (does not violate
        BCNF)
    \item course id \textrightarrow title (violates BCNF)
\end{itemize}

What superkeys do we have?

Answer: any superset of our only key, which is \{course id, section\}.

\begin{itemize}
    \item Let's pick our first violating FD to work with first: instructor
        \textrightarrow office
    \item Next, expand the RHS as much as possible (we want the closure of
        instructor)

        \begin{center}
            instructor \textrightarrow \{instructor, office, email\}
        \end{center}
    \item Now we decompose into two new tables, shown in the next slide:
        \begin{itemize}
            \item $R_1 = \pi_{instructor,\ office,\ email}(R)$
            \item $R_2 = \pi_{instructor,\ course\ id,\ section,\ title}(R)$
        \end{itemize}
    \item Now table $R_1$ is in BCNF (note that this is not guaranteed by the
        algorithm -- we could have had other violating FDs)
    \item Table $R_2$ has a violating FD though: course id \textrightarrow title
    \item Decomposition of $R_2$ via course id \textrightarrow title:
        \[\text{course\_id}^+ = \{\text{course\_id, title}\}\]
    \item Decompose into
        \[R_3 = \pi_{\text{course\_id, title}}(R_2)\]
        \[R_4 = \pi_{\text{instructor, course\_id, title}}(R_2)\]
    \item Done!
        \begin{itemize}
            \item Three tables remain: $R_1$, $R_3$, $R_4$
            \item All non-essential redundancy has been removed
            \item Each table now represents a fundamental entity
                \begin{itemize}
                    \item $R_1$ = instructor info
                    \item $R_3$ = course info
                    \item $R_4$ = section info
                \end{itemize}
        \end{itemize}
\end{itemize}

\section{Review}
\begin{itemize}
    \item \textbf{Superkey}: a subset $X$ of attributes of $R$: no two tuples of
        $R$ will ever agree on $X$.
    \item \textbf{FD (Functional Dependency)}: $X \rightarrow Y$ on $R$:
        whenever two tuples agree on $X$, they must also agree on $Y$.
    \item if $a$ is a superkey $a \rightarrow \{a, b, \dots\}$
\end{itemize}

\section{Correctness of Decomposition}
Two requirements for correct decomposition so that we can recover original
relation from decomposition using natural joins

\begin{enumerate}
    \item natural join of decomposition must contain all attributes of the
        original relation
    \item lossless join property: natural join of decomposition relations
        results in exactly the same tuples we had before decomposition
\end{enumerate}

\section{Multivalued Dependencies (MVD)}
\textbf{Def:} An MVD $X \twoheadrightarrow Y$ exists on a relation $R$ if
whenever there are two tuples $t_1$ and $t_2$ which agree on attribute $X$, then
there also exists a tuple $t_3$ (which could be $t_1$ or $t_2$) such that the
following are true:

\begin{enumerate}
    \item $t_3[x] = t_1[x] = t_2[x]$
    \item $t_3[y] = t_2[y]$
    \item $t_3[z] = t_1[z]$ where $z$ is everything in $R$ \textit{not} in $X$
        and $Y$.
\end{enumerate}

By symmetry, there must also exist $t_4$ with $t_4[x] = t_1[x]$, $t_4[y] =
t_4[y]$, $t_4[z] = t_2[z]$.

\end{multicols*}

\end{document}
