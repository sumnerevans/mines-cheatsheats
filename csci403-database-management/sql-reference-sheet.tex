\documentclass{cheatsheet}

\course{CSCI 406}
\cheatsheetname{SQL Reference Sheet}

\begin{document}

\begin{multicols*}{2}
    \section{Overview}
    \begin{minted}{postgresql}
        SELECT <attributes>
          FROM <tablenames>
         WHERE <condition>;
    \end{minted}

    \begin{itemize}
        \item \texttt{<attributes>} - the attributes to select
        \item \texttt{<tablenames>} - the table names to select from
        \item \texttt{<condition>} is a boolean expression on the attributes of
            the table

    \end{itemize}

    \section{\texttt{WHERE} Condition}
    \begin{itemize}
        \item \texttt{<>} $\equiv$ not equals

        \item other operators include \texttt{<,>,<=,>=, BETWEEN}.
            (\texttt{BETWEEN} is inclusive.)
            \begin{minted}{postgresql}
                        ... WHERE max_credits BETWEEN 3 AND 6;
            \end{minted}

        \item compound expressions: \texttt{AND, OR, NOT}

        \item Testing for \texttt{NULL}: must use \texttt{IS NULL} or
            \texttt{IS NOT NULL}

        \item \texttt{LIKE} and \texttt{NOT LIKE}
            \begin{minted}{postgresql}
                ... WHERE instructor LIKE 'Paint%';
            \end{minted}

        \item \texttt{IN}
            \begin{minted}{postgresql}
               ... WHERE x IN (1, 2, 3);
            \end{minted}
    \end{itemize}

    \section{Select Statements}
    \subsection{Selecting Expressions on Attributes}
    \begin{minted}{postgresql}
    SELECT 42 / 13 + 12; -- selects 15 (integer math)
    SELECT a || ' ' || b || ' ' || c FROM foo; -- string concatenation
    SELECT substring(a FROM 1 FOR 4) FROM foo; -- first four characters
    \end{minted}

    \subsection{Names and Aliasing}
    \texttt{AS} - used for renaming

    \begin{minted}{postgresql}
    SELECT substring(foo FROM 1 FOR 4) as f, bar as b FROM baz;
    \end{minted}

    \subsection{Schemas}
    \begin{minted}{postgresql}
        -- given "project1" in cpainter and "project1" in public
        SELECT * FROM public.project1; -- selets the public one
    \end{minted}

    \subsection{Misc}
    \begin{minted}{postgresql}
    SELECT count(*) FROM mines_courses_meetings;
    SELECT DISTINCT a1, a2, a3 ...
    \end{minted}

    \section{Joins}
    \begin{minted}{postgresql}
        SELECT * FROM A, B WHERE A.x = B.x;
        SELECT * FROM A JOIN B ON B.x = A.x; -- using join syntax
    \end{minted}

    \section{Order By}
    \begin{minted}{postgresql}
    ... ORDER BY attr DESC/ASC
    \end{minted}

    \section{Table Creation}
    \begin{minted}{postgresql}
        CREATE TABLE [schema_name.]table_name
        (
            attribute1 type1 NOT NULL, -- you can add constraints
            attribute2 type2 PRIMARY KEY,
            attribute3 type3
        )
    \end{minted}

    \section{Examples}
    \begin{minted}{postgresql}
        SELECT mc.instructor, mc.course_id,
               mef.office, mef.email
          FROM mines_courses as mc, mines_eecs.faculty AS mec
         WHERE mc.instructor = mef.name;
    \end{minted}
    \begin{minted}{postgresql}
        SELECT * FROM foo WHERE bar = 3 ORDER BY alpha, beta, gamma;
    \end{minted}


\end{multicols*}

\end{document}
