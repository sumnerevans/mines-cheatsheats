\documentclass{../cheatsheet}

\course{CSCI 341 --- Computer Organization}
\cheatsheetname{Cache}

\begin{document}

\begin{multicols*}{2}
    \section{Memory Hierarchy}
    \begin{tabular}{l c}
        Processor (registers) & Fast, small, expensive \\
        Cache & $\downarrow$ \\
        Main Memory & $\downarrow$ \\
        Disk & Slow, large, cheap
    \end{tabular}

    \begin{itemize}
        \item \textbf{Hit}: data appears in some block in the upper levels of
            memory (closer to the processor)
            \begin{itemize}
                \item \textbf{Hit Rate}: the fraction of memory access found in
                    the upper level
                \item \textbf{Hit Time}: the time to access the upper level \\
                    = memory access time + time to determine hit/miss
            \end{itemize}

        \item \textbf{Miss}: data needs to be retrieved from a block in the
            lower level of memory.
            \begin{itemize}
                \item \textbf{Miss Rate}: 1 - (hit rate)
                \item \textbf{Miss Penalty} = time to replace block in upper
                    level + time to deliver block to processor
            \end{itemize}

        \item \textbf{average access time}\\
            = hit time $\times$ hit rate + miss penalty $\times$ miss rate

        \item \textbf{average memory access time (AMAT)}\\
            = hit time + miss penalty $\times$ miss rate
    \end{itemize}

    \section{Cache}
    \subsection{Direct-Mapped Cache}
    \begin{itemize}
        \item Each memory address is associated with one possible \textit{block}
            within the cache
        \item \textbf{Block}: the unit of transfer between cache and memory
        \item Address in Direct-Mapped Cache\\
            \begin{tabular}{|c c c | c | c|}
                \hline
                t & a & g & idx & o \\
                \hline
            \end{tabular}
            \begin{itemize}
                \item \textbf{tag}: to check if have correct block
                \item \textbf{index}: to select block (which "row")
                \item \textbf{byte offset} within block
            \end{itemize}
        \item larger cache blocks take better advantage of spacial locality
        \item every block has a \textit{valid bit} determines whether anything
            stored in that row
    \end{itemize}

    \subsubsection{Example}
    16KB of data in direct-mapped cache with 4 word blocks. Determine size of
    tag, index and offset fields (given 32-bit architecture)
    \begin{itemize}
        \item \textbf{index}: \# blocks = 16 KB / $2^4$ = $2^{10}$ B
            $\rightarrow$ 10 bits
        \item \textbf{byte offset}: 4 words = 16 bytes = $2^4$ bytes
            $\rightarrow$ 4 bits
        \item \textbf{tag}: 32 - 4 - 10 = 18 $\rightarrow$ 18 bits
    \end{itemize}

    \subsubsection{Impact of Large Block Size}
    \begin{itemize}
        \item \textbf{benefit}: reduce miss rate (takes advantage of spacial
            locality)
        \item \textbf{drawbacks}: larger miss penalty (takes longer to load new
            block from next level) and possibly a higher miss rate
    \end{itemize}

    \subsection{Fully Associative Cache}
    \begin{itemize}
        \item no rows, basically a hash map
        \item \textbf{benefits}: no conflicts among different memory addresses
        \item \textbf{drawbacks}: need hardware comparator for every single
            entry: this is infeasible
    \end{itemize}

    \subsection{N-Way Set Associative Cache}
    \begin{itemize}
        \item memory address fields: tag (same as before), offset (same as
            before), index (points us to correct row (called a set in this
            case))
        \item cache is direct-mapped with respect to sets, each set is set
            associative
        \item has best of direct-mapped and fully associative
    \end{itemize}

    \subsection{Examples: Number of Bits Needed}
    64 KB of data and one word blocks\\
    64 KB = 16 K words = $2^{14}$ words = $2^{14}$ blocks
    \begin{itemize}
        \item \textbf{direct-mapped cache}\\
            block size = 4 bytes $\rightarrow$ offset size = 2 bits\\
            \# sets = \# blocks = $2^{14}$ $\rightarrow$ index size = 14 bits\\
            tag size = 32 - 14 - 2 = 16 bits\\
            bits/block = data + tag + valid = 32 + 16 + 1 = 49\\
            bits/cache = \# blocks $\times$ bit/block = $2^{14} \times 49 = 98$
            KB

        \item \textbf{4-way set associative}\\
            block size = 4 bytes $\rightarrow$ offset size = 2 bits\\
            \# sets = \# blocks/4 = $2^{14}/4 = 2^{12}$ $\rightarrow$ index size
            = 12 bits\\
            tag size = 32 - 12 - 2 = 18 bits\\
            bits/block = data + tag + valid = 32 + 18 + 1 = 51\\
            bits/cache = \# blocks $\times$ bit/block = $2^{14} \times 51 = 102$
            KB

        \item \textbf{increase associativity $\rightarrow$ increase bits in
            cache}

        \item \textbf{8 word blocks}\\
            64 KB = $2^{14}$ words = $2^{14}/8$ blocks = $2^{11}$ blocks\\
            block size = 4 words = 32 bytes $\rightarrow$ offset size = 5 bits\\
            \# sets = \# blocks = $2^{11} \rightarrow$ index size = 11 bits\\
            tag size = address - index - offset = 32 - 11 - 5 = 18 bits\\
            bits/block = data + tag + valid = $8 \times 32 + 16 + 1 = 273$bits\\
            bits/cache = \# blocks $\times$ bit/block = $2^{11} \times273=68.25$
            KB

        \item \textbf{increase block size $\rightarrow$ decrease bits in cache}
    \end{itemize}

    \subsection{Accessing a Cache}
    \begin{itemize}
        \item \textbf{cache hit}: cache block is valid and contains proper
            address, read the desired word
        \item \textbf{cache miss}: nothing in cache in appropriate block, fetch
            from memory
        \item \textbf{cache miss, block replacement}: wrong data is in cache at
            appropriate block, so discard it and fetch desired data from memory
        \item \textbf{Types of Cache Misses}
            \begin{itemize}
                \item Compulsory Miss: occur when program first starts
                \item Conflict Misses: occurs when two addresses map to the same
                    cache location (problem in direct-mapped cache)
                \item Capacity Misses: miss that occurs because the cache has
                    limited size (problem in fully associative caches)
            \end{itemize}
    \end{itemize}

    \subsection{Block Replacement}
    \begin{itemize}
        \item if there are any locations with valid bit of 0, usually write over
            that
        \item if all possible locations already have valid block, choose
            replacement strategy
            \begin{itemize}
                \item random (simple to implement)
                \item least-recently used (LRU) (expensive)
            \end{itemize}
    \end{itemize}

    \subsection{Mulit-Level Cache Hierarchy}
    Given 2 levels of cache:\\
    AMAT = L1 hit time + L1 miss rate $\times$ \textbf{L1 miss penalty}\\
    L1 miss penalty = L2 hit time + L2 miss rate $\times$ L2 miss penalty

    \section{Examples}
    \subsection{Average Memory Address Time (AMAT)}
    Assume: hit time = 1 cycle, miss rate = 5\%, miss penalty = 20 cycles\\
    AMAT = $1 + 0.05 \times 20 = 1 + 1 cycles = 2 cycles$
\end{multicols*}

\end{document}
